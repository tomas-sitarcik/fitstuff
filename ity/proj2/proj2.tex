\documentclass[11pt, twocolumn, a4paper]{article}
\usepackage[utf8]{inputenc}
\usepackage[left=1.4cm, text={18.2cm, 25.2cm}, top=2.3cm]{geometry}
\usepackage[czech]{babel}
\usepackage[IL2]{fontenc}
\usepackage[unicode]{hyperref}
\usepackage{chngpage}
\usepackage{hyperref}
\usepackage{footnotebackref}
\usepackage{times}
\usepackage{amsthm}
\usepackage{amssymb}
\usepackage{amsmath}
%\usepackage{amsfonts}
\theoremstyle{definition}
\newtheorem{definition}{Definice}
\newtheorem{sentence}{Věta}
 
\title{Typografie a publikování - 2. projekt}
\author{Tomáš Sitarčík}
 
\begin{document}
 
\begin{titlepage}
    \begin{center}
 
        \Huge
        \textsc{Vysoké učení technické v~Brně\\Fakulta informačních technologií} \\
        \vspace{\stretch{0.382}}
        \LARGE
        Typografie a publikování -- 2. projekt\\Sazba dokumentů a matematických výrazů
        \vspace{\stretch{0.61}}
      
        {\LARGE 2023 \hfill Tomáš Sitarčík (xsitar06)}
    
    \end{center}
\end{titlepage}


\newpage
\twocolumn

$$\overset{iid}{\sim}$$

\section*{Úvod}
V~této úloze si vyzkoušíme sazbu titulní strany, matematických vzorců, prostředí a dalších 
textových struktur obvyklých pro technicky zaměřené texty například Definice 1 nebo rovnice
(3) na straně 1. Pro vytvoření těchto odkazů používáme kombinace příkazů
\verb|\label, \ref, \eqref a \pageref|. Před odkazy patří nezlomitelná mezera. Pro
zvýrazňování textu jsou zde několikrát použity příkazy \verb|\verb| a \verb|\emph|.

Na titulní straně je použito prostředí \verb|titlepage| a sázení nadpisu podle optického
středu s~využitím přesného zlatého řezu. Tento postup byl probírán na přednášce. Dále
jsou na titulní straně použity čtyři různé velikosti písma a~mezi dvojicemi řádků textu
je použito odřádkování se zadanou relativní velikostí 0,5\,em a 0,4\,em 
\footnote[1]{Nezapomeňte použít správný typ mezery mezi číslem a jednotkou.}.

\section{Matematický text}
V~této sekci se podíváme na sázení matematických symbolů a výrazů v~plynulém textu pomocí
prostředí math. Definice a věty sázíme pomocí příkazu \verb|\newtheorem| s~využitím balíku
amsthm. Někdy je vhodné použít konstrukci \verb|${}$| nebo \verb|\mbox{}|, která říká, že
(matematický) text nemá být zalomen.
\begin{definition}
    \emph{Zásobníkový automat (ZA) je definován jako sedmice tvaru}
    ${A=(Q,\Sigma, \Gamma , \delta , q_0 , Z_0 , F)}$ \emph{kde:}
    \begin{itemize}
        \item ${Q}$ \emph{je konečná množina vnitřních (řídicích) stavů,}
        \item ${\Sigma}$ \emph{je konečná vstupní abeceda,}
        \item ${\Gamma}$ \emph{je konečná zásobníková abeceda,}
        \item ${\delta}$ \emph{je} přechodová funkce
        ${Q \times (\Sigma \cup \{\epsilon\})\times \Gamma \rightarrow 2^{Q \times \Gamma^\ast}}$
        \item ${q_0 \in Q}$ \emph{je počáteční stav, ${Z_0 \in \Gamma}$ je startovací symbol
        zásobníku a ${F \subseteq Q}$ je množina koncových stavů.}
    \end{itemize}
    \par Nechť ${P=(Q,\Sigma, \Gamma , \delta , q_0 , Z_0 , F)}$ je ZA. \emph{Konfigurací}
    nazveme trojici ${(q, w, \alpha) \in Q \times \Sigma^\ast \times \Gamma^\ast}$
    , kde \emph{q} je aktuální stav vnitřního řízení, $w$ je dosud nezpracovaná část vstupního řetězce a
    ${\alpha = Z_{i_{1}} Z_{i_{2}} \dots Z_{i_{k}}}$ je obsah zásobníku.
\end{definition}

\subsection{Podsekce obsahující definici a větu}
\begin{definition}
    Řetězec $w$ nad abecedou $\Sigma$ je přijat ZA \emph{A~jestliže} ${(q_0, w, Z_0)}$
    $\overset{\ast}{\underset{A}{\vdash}}$ ${(q_F, \epsilon, \gamma)}$ \emph{pro nějaké}
    ${\gamma \in \Gamma^\ast}$ \emph{a} ${q_F \in F}$. \emph{Množina}
    ${L(A) = \{w~|~w}$ \emph{je přijat} \emph{ZA}~${ A\} \subseteq \Sigma^\ast}$ je jazyk přijímaný ZA \emph{A}.
\end{definition}
\begin{sentence}
    \emph{Třída jazyků, které jsou přijímány ZA, odpovídá} bezkontextovým jazykům.
\end{sentence}

\section{Rovnice}
Složitější matematické formulace sázíme mimo plynulý text pomocí prostředí \verb|displaymath|.
Lze umístit i několik výrazů na jeden řádek, ale pak je třeba tyto vhodně oddělit, například
příkazem \verb|\quad|.
\begin{displaymath}
    1^{2^{3}} \neq \Delta^{1}_{\Delta^{2}_{\Delta^{3}}} \quad
    y^{11}_{22} - \sqrt[\leftroot{-1}\uproot{1}9]{x + \sqrt[\leftroot{-1}\uproot{1}7]{y}} \quad
    x > y_1 \leq y^2
\end{displaymath}
V~rovnici (2) jsou využity tři typy závorek s~různou explicitně definovanou velikostí.
Také nepřehlédněte, že nasledující tři rovnice mají zarovnaná rovnítka, a použijte
k~tomuto účelu vhodné prostředí.
\begin{align}
        - \cos^2 \beta = & \frac{\frac{\frac{1}{x} + \frac{1}{3}}{y} + 1000}
         {\overset{8}{\underset{j=2}{\prod}}~q_j} &\\
        \biggl( \Bigl\{ b \star [3 \div 4] \circ a \Bigl\}^{\frac{2}{3}} \biggl) = & \log_{10}x & \\
        \int_{a}^{b} f(x)\,dx = & \int_{c}^{d} f(y)\,dy 
\end{align}
V~této větě vidíme, jak vypadá implicitní vysázení limity ${\lim_{m\to\infty}f(m)}$ 
v~normálním odstavci textu. Podobně je to i s~dalšími symboly jako ${\bigcup_{N \in \mathcal{M}}N}$ či
${\sum_{i=1}^{m} x^{2}_{i}}$. S~vynucením méně úsporné sazby příkazem \verb|\limits| budou vzorce 
vysázeny v~podobě ${{\lim\limits_{m\to\infty}}} f(m)$ a ${\sum\limits_{i=1}^{m}x^{4}_{i}}$.
\section{Matice}
Pro sázení matic se velmi často používá prostředí array a závorky (\verb|\left|, \verb|\right|).
\[
    \mathbf{B} =
    \begin{vmatrix}

        b_{11} & b_{12} & \cdots & b_{1n} \\
        b_{21} & b_{22} & \cdots & b_{2n} \\
        \vdots & \vdots & \ddots & \vdots \\
        b_{m1} & b_{m2} & \cdots & b_{mn} 

    \end{vmatrix}
    =
    \begin{vmatrix}

        t & u~\\
        v~& w 
    
    \end{vmatrix}
    =
    tw - uv
\]
\[
    \mathbb{X} = \mathbf{Y} \Longleftrightarrow 
    \begin{bmatrix}
            & \Omega + \Delta & \hat{\psi } \\
            \vec{\pi} & \omega & \\
    \end{bmatrix}
    \neq 42
\]
\par Prostředí \verb|array| lze úspěšně využít i jinde, například na pravé straně následující
rovnice. Kombinační číslo na levé straně vysázejte pomocí příkazu \verb|\binom|.
\[
    \binom{n}{k} = 
    \begin{cases}
        0 & \textrm{pro }k<0 \\
        \frac{n!}{k!(n-k)!} & \textrm{pro }0 \leq k \leq n \\
        0 & \textrm{pro }k>0
    \end{cases}
\] 
\end{document}
