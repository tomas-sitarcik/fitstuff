\documentclass[11pt, a4paper, hyphens]{article}
\usepackage[utf8]{inputenc}
\usepackage[left=2cm, text={17cm, 24cm}, top=3cm]{geometry}
\usepackage[czech, english]{babel}
\usepackage[IL2]{fontenc}
\usepackage{times}
\usepackage{amsmath}
\usepackage[backref=section]{hyperref}

\title{Typografie a publikování --- 4.\, projekt}
\author{Tomáš Sitarčík}
\bibliographystyle{czechiso}

\begin{document}
\begin{titlepage}
    \begin{center}

        \textsc{\Huge Vysoké učení technické v~Brně\\
            \LARGE Fakulta informačních technologií\\}
        \vspace{\stretch{0.382}}
        \LARGE
        Typografie a publikování --- 4. projekt\\\Huge Sazba matematických výrazů\\v prostředí a víc \LaTeX
        \vspace{\stretch{0.618}}

        {\Large \today \hfill Tomáš Sitarčík}

    \end{center}
\end{titlepage}
\newpage

\section{Úvod}
\LaTeX je jedním z~nejpoužívanějších~\cite{MoorheadAltheaV.2021ILuc} a nejflexibilnějších prostředí na sázení matematických výrazů.
Tento dokument obsahuje základy i některé pokročilé způsoby jejich sazbu a další užitečné nástroje pro práci s~\LaTeX em

\section{Matematický mód}
Pokud chceme zapsat nějaký matematický výraz tak ho musíme ohraničit speciálnímím znakem \verb|$|.
Toto je ten nejjednodušší způsob jak zapnout matematický mód, další způsoby budou zmíněny v~dalších kapitolách.
V~Matematickém módu pak můžeme používat speciální příkazy jako například \verb|\times| $\rightarrow \times$ nebo
\verb|\frac{a}{b}| $\rightarrow \frac{a}{b}$. Jakýkoli text v~prostředí matematického módu je také zobrazený v~matematické fontu,
například \verb|text| vypadá v~matematickém módu jako $text$.~\cite{SyropoulosApostolos1900DtuL}

\section{Další způsoby vstupu do matematického módu}
Matematický mód má více druhů. Při použití dvou \verb|$| na každé straně, tak se výsledný výraz zobrazí na vlastním řádku
s~odsazením od okolního textu. Tento mód se také dá vyvolat pomocí \verb|\[ \]|~\cite{MatthewsDavid2019Cbei}, prostředí \verb|equation| a další.
Příklad jak vypadá zkompilované použití tohoto módu se nachází v~další kapitole.

\section{Odsazení symbolů}
U~rovnic o~více neznámých je nahrazení proměnné která má exponent roven nule, za prázdné místo dobré pro čitelnost.
Pomocí prostředí \verb|alignat*{n}|(kde \verb|n| je počet řádků) je možno vysázet několik rovnic podobným způsobem
jako tabulku.
Využitím \verb|alignat| tedy můžeme vytvořit zarovnanou soustavu rovnic.

\verb|\begin{alignat*}{2}| \\
\verb|      (A + B C)x &{}+{} &C &y = 0| \\
\verb|      Ex &{}+{} &(F + G)&y = 23.| \\
\verb|   \end{alignat*}|~\cite{GratzerGeorgeA.2007MmiL} \\
\quad Po kompilaci:
\begin{alignat*}{2}
    (A~+ B C)x & + & C       & y = 0,  \\
    Ex         & + & (F + G) & y = 23.
\end{alignat*}

\section{Řecké znaky}
Řecké znaky jsou v~matematice velice časté. Ve většině případů vypadá příkaz následovně \verb|\[phoneticky zapsané písmeno]| takže
například \verb|\phi| pro fí. Jsou ale i záludnější znaky jako \verb|\varphi|~\cite{SebekMichal2007Idsp} a další.

\section{Matematické výrazy na webových stránkách}
\LaTeX bohužel není možné tradičně použít k~zobrazení celé webové stránky. Pro sázení na webu se hodí JavaScriptová knihovna Katex, která je schopná
zobrazi vzorce v~jazyku \TeX~\cite{SimovaHana2018Zmvd}.

\section{Výsledky externího kódu v~\LaTeX u}
Při výzkumu například SAT solverů se může hodit možnost přímo z~\LaTeX u zapínat programy a ve výsledném pdf souboru zahrnout jejich výstup.
S~použitím python balíku \texttt{talk2stat} je možné právě to~\cite{BarHaim2021RSwL}.

\section{Overleaf}
Overleaf je velmi užitečný moderní editor pro \LaTeX. Běží přímo v~prohlížeči (překlad probíhá na serveru) takže
ůžete pracovat kdekoli. Má i další funkce jako kolaborace několika uživatelů na jednom projektu~\phantomsection\cite{NovotnyVit2021OCOL}.

\section{Alternativný možnosti tvorby výrazů v~\LaTeX u}
Existují různé způsoby jak si zlehčit zadávání výrazů v~\LaTeX u. Většina těchto způsobů funguje v~rámci GUI, jeden ze zvláštnějších způsobů
je pomocí rozpoznávání textu počítačem. To ale má svá úskalí.
\begin{quotation}
    \uv{\textit{
    V~našem rozpoznávacím systému předpokládáme že
    odstavce nebo dokumenty jsou psány na řádkovaném papíru,
    řádek po řádku. Také je každý řádek rozdělen
    na 3 pruhy pomocí 4 referenčních linke.
    }}~\cite{1394451}
\end{quotation}
Jak jde vivodit z~citace, aby počítač text správně rozpoznal, tak musím být napsaný specifickým způsobem na speciální papír.

%bibliography file "citations.bib"

\bibliography{citations}

\end{document}
